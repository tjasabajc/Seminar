\documentclass{beamer}

\usepackage[slovene]{babel}
\usepackage{amsfonts,amssymb}
\usepackage[utf8]{inputenc}
\usepackage{lmodern}
\usepackage[T1]{fontenc}



\usetheme{Copenhagen}
\usecolortheme{crane}

\def\N{\mathbb{N}} % mnozica naravnih stevil
\def\Z{\mathbb{Z}} % mnozica celih stevil
\def\Q{\mathbb{Q}} % mnozica racionalnih stevil
\def\R{\mathbb{R}} % mnozica realnih stevil
\def\C{\mathbb{C}} % mnozica kompleksnih stevil


\def\qed{$\hfill\Box$}   % konec dokaza
\newtheorem{izrek}{Izrek}
\newtheorem{trditev}{Trditev}
\newtheorem{posledica}{Posledica}
\newtheorem{lema}{Lema}
\newtheorem{definicija}{Definicija}
\newtheorem{pripomba}{Pripomba}
\newtheorem{primer}{Primer}
\newtheorem{zgled}{Zgled}
\newtheorem{zgledi}{Zgledi uporabe}
\newtheorem{zglediaf}{Zgledi aritmetičnih funkcij}
\newtheorem{oznaka}{Oznaka}

\title{Delitev brez zavisti}
\author{Tjaša Bajc}
\date{19.\ maj 2017}

\begin{document}


%%%%%%%%%%%%%%%%%%%%%%%%%%%%%%%%%%%%%%%%%%%%%%%%%%%%%%%%%%%%%%%%%%%%%

\begin{frame}
\titlepage
\end{frame}

%%%%%%%%%%%%%%%%%%%%%%%%%%%%%%%%%%%%%%%%%%%%%%%%%%%%%%%%%%%%%%%%%%%%%
\section{Uvod}
\subsection{Začetni privzetki}

\begin{frame}

Potrebno je privzeti naslednje:

\begin{itemize}
\item za vse igralce je mera celote $C$ enaka 1, $\mu_i(C) = 1, i \in \{1, 2, \ldots, n\}$;
\item kosov z enako mero med seboj ne razlikujemo; 
\item vsak kos je mogoče nadalje razdeliti na $k$ enako velikih kosov;
\item za vsaka kosa $A$ in $B$ je mogoče kos $A$ obrezati tako, da bo $\mu(A) = \mu(B)$, ali obratno.							
\end{itemize}

\end{frame}

%%%%%%%%%%%%%%%%%%%%%%%%%%%%%%%%%%%%%%%%%%%%%%%%%%%%%%%%%%%%%%%%%%%%%
\subsection{Različne delitve}
\begin{frame}

Kaj pomeni ``pravična'' delitev?

\begin{definicija}
Naj bo $C$ množica, $\mu_1, \mu_2, \ldots, \mu_n$ po vrsti mere igralcev $1, 2, \ldots, n$ in $\{P_1, P_2, \ldots, P_n\}$ delitev celote $C$ na $n$ delov, kjer igralcu $i$ pripada kos $P_i$. Delitev je:

\begin{itemize}
\item \textbf {\em sorazmerna}, če za vsak $i$ velja $\mu_i(P_i) \geq 1/n$;
\item \textbf {\em brez zavisti}, če za vsaka $i, j$ velja $\mu_i(P_i) \geq \mu_i(P_j)$;
\item \textbf {\em super brez zavisti}, če za vsaka $i, j, i \neq j,$ velja $\mu_i(P_i) > 1/n$ in hkrati $\mu_i(P_j) < 1/n$ .								
\end{itemize}

\end{definicija}

\end{frame}

%%%%%%%%%%%%%%%%%%%%%%%%%%%%%%%%%%%%%%%%%%%%%%%%%%%%%%%%%%%%%%%%%%%%%
\subsection{Protokoli}
\begin{frame}

Protokol je sestavljen iz

\begin{itemize}
\item pravil,
\item strategij,
\item dokaza.
\end{itemize}

\begin{zgled}
Metoda ``Reži in izberi'' za delitev med dva igralca.
\begin{enumerate}

\item Prvi igralec razdeli torto na dva dela \textcolor{gray}{(ki sta zanj enako velika)}.

\item Igralca si razdelita kose. 

{\em Vrstni red izbiranja: Drugi igralec, prvi igralec}

\end{enumerate}

{\em Komentar:} Ta delitev je {\em sorazmerna\/} in hkrati {\em brez zavisti\/}. Prvi igralec je dobil natančno $1/2$, drugi pa kvečjemu več. Če namreč njuni meri sovpadata, $\mu_1 = \mu_2$, je prav tako dobil natančno $1/2$, v nasprotnem primeru pa si lahko izbere večjega od obeh kosov.


\end{zgled}

\end{frame}


%%%%%%%%%%%%%%%%%%%%%%%%%%%%%%%%%%%%%%%%%%%%%%%%%%%%%%%%%%%%%%%%%%%%%
\section{Sorazmerna delitev}
\subsection{Trije igralci}
\begin{frame}


\frametitle{Protokol za sorazmerno delitev med tri igralce}
\begin{enumerate}

\item Prvi igralec razdeli torto na tri dele \textcolor{gray}{(ki so zanj enako veliki)}.

\item Drugi igralec lahko ne naredi ničesar \textcolor{gray}{(če vsaj dva od treh kosov vrednoti kot ali enaka $1/3$ celote)}

ali pa dva kosa označi kot ``slaba''. \textcolor{gray}{(To stori, če sta vsak posebej zanj vredna strogo manj kot $1/3$ celote.)}

\item Če drugi igralec v prejšnjem koraku ni storil nič, smo končali in si igralci izberejo kose torte.

\textsl{Vrstni red izbiranja: Tretji igralec, drugi igralec, prvi igralec.}

\end{enumerate}
\end{frame}
%%%%%%%%%%%%%%%%%%%%%%%%%%%%%%%%%%%%%%%%%%%%%%%%%%%%%%%%%%%%%%%%%%%%
\begin{frame}
\frametitle{\textcolor{gray}{Protokol za sorazmerno delitev med tri igralce}}
\begin{enumerate}
\setcounter{enumi}{3}

\item Sicer (če je v 2. koraku drugi igralec dva kosa označil kot ``slaba'') ponovimo drugi korak, le da je zdaj na potezi tretji igralec. Oznake drugega igralca prekrijemo, tretji igralec jih ne upošteva.

\item Če v prejšnjem koraku tretji igralec ni storil nič, smo končali in si igralci izberejo kose torte.

\textsl{Vrstni red izbiranja: Drugi igralec, tretji igralec, prvi igralec.}


\item Če je tretji igralec v četrtem koraku dva kosa označil kot ``slaba'', mora prvi igralec vzeti kos, ki sta ga tako drugi kot tretji igralec določila za ``slabega''.

\item Preostala kosa torte ponovno združimo. To novo torto si drugi in tretji igralec razdelita po metodi ``Reži in izberi''.

\end{enumerate}

\end{frame}

%%%%%%%%%%%%%%%%%%%%%%%%%%%%%%%%%%%%%%%%%%%%%%%%%%%%%%%%%%%%%%%%%%%%%
\subsection{Splošna rešitev}
\begin{frame}
\frametitle{Protokol za sorazmerno delitev v splošnem}

\begin{enumerate}

\item Prvi igralec od celote odreže kos $P_1$ \textcolor{gray}{(ki je zanj vreden natančno $1/n$)}.

\item Drugi igralec lahko ne naredi ničesar \textcolor{gray}{(če je zanj kos $P_1$ vreden manj od $1/n$)}
ali pa kos $P_1$ obreže \textcolor{gray}{(tako, da bo obrezan kos vreden natančno $1/n$)}.

Kos preimenujemo v $P_2$. Tisto, kar je drugi igralec odrezal stran, shranimo na posebno mesto za ostanke.

\item Za igralce od tretjega do zadnjega po vrsti naredimo naslednje: Igralec $i$ vzame kos $P_{i-1}$ od svojega predhodnika in ravna enako kot drugi igralec v prejšnjem koraku. Kos preimenujemo v $P_i$ in podamo naprej. %, morebitne ostanke pa shranjujemo na enem mestu.

\end{enumerate}
\end{frame}
%%%%%%%%%%%%%%%%%%%%%%%%%%%%%%%%%%%%%%%%%%%%%%%%%%%%%%%%%%%%%%%%%%%%
\begin{frame}
\frametitle{\textcolor{gray}{Protokol za sorazmerno delitev v splošnem}}
\begin{enumerate}
\setcounter{enumi}{3}

\item Kos $P_n$ dobi zadnji igralec, ki je obrezal prejeti kos, ali pa prvi igralec, če ga ni nihče obrezal.

\item Vse obrezane koščke združimo s preostankom začetne torte in tako dobimo novo torto, ki jo želimo sorazmerno razdeliti med $n - 1$ ljudi. 

Postopek ponavljamo, dokler nam ne ostaneta samo dva igralca.

\item Za zadnja igralca sledimo metodi ``Reži in izberi''.

\end{enumerate}

\end{frame}

%%%%%%%%%%%%%%%%%%%%%%%%%%%%%%%%%%%%%%%%%%%%%%%%%%%%%%%%%%%%%%%%%%%%%
\section{Delitev brez zavisti}
\subsection{Trije igralci}
\begin{frame}
\frametitle{Protokol za delitev brez zavisti za tri igralce}

\begin{enumerate}

\item Prvi igralec razdeli torto na tri dele \textcolor{gray}{(ki so zanj enako veliki)}.

\item Drugi igralec lahko ne naredi ničesar \textcolor{gray}{(če vrednoti vsaj dva od treh kosov kot enako velika in največja)}
ali pa obreže enega od kosov. \textcolor{gray}{(Obreže največjega, in sicer tako, da bo po obrezovanju enako velik kot drugi največji.)}

Če je obrezal kos, odrezani del shranimo na mesto za ostanke in ga imenujemo $O$.

\item Igralci si izberejo kose torte, ostankov ne upoštevamo. \textcolor{gray}{(Vsak izbere največji kos ali enega od enako velikih največjih kosov.)} Zahtevamo, da drugi igralec vzame obrezani kos, če ga ni pred njim izbral tretji igralec.

\textsl{Vrstni red izbiranja: Tretji igralec, drugi igralec, prvi igralec.}

\end{enumerate}
\end{frame}
%%%%%%%%%%%%%%%%%%%%%%%%%%%%%%%%%%%%%%%%%%%%%%%%%%%%%%%%%%%%%%%%%%%%
\begin{frame}
\frametitle{\textcolor{gray}{Protokol za delitev brez zavisti za tri igralce}}
\begin{enumerate}
\setcounter{enumi}{3}

\item Če drugi igralec ni obrezal kosa, smo končali, saj ni ostanka in smo že dosegli {\em delitev brez zavisti}. V nasprotnem primeru pa sta si drugi in tretji igralec razdelila en obrezan in en neobrezan kos. Tistega, ki je dobil neobrezan kos, imenujmo ``rezalec'', drugega pa ``ne-rezalec''. ``Rezalec'' sedaj razdeli $O$ na tri dele.

\item Igralci si izberejo kose, na katere je bil razdeljen ostanek. \textcolor{gray}{(Vsak izbere največji kos ali enega od enako velikih največjih kosov.)}

\textsl{Vrstni red izbiranja: Ne-rezalec, prvi igralec, rezalec.}

\end{enumerate}

\end{frame}

%%%%%%%%%%%%%%%%%%%%%%%%%%%%%%%%%%%%%%%%%%%%%%%%%%%%%%%%%%%%%%%%%%%%%
\subsection{Splošna rešitev - skica}
\begin{frame}

Kako pridemo do protokola za splošen $n$?

\begin{itemize}

\item \textbf{Obrezovanje} 

S tem dosežemo, da je več kosov enako velikih in največjih.

\item \textbf{Delne razdelitve} 

Če so vmesne razdelitve {\em brez zavisti}, je taka tudi končna razdelitev.

\item \textbf{Ostanki in vprašanje končnosti} 

Obstaja zgornja meja za število korakov (pojem ``nepreklicne prednosti'').

\end{itemize}

\end{frame}

%%%%%%%%%%%%%%%%%%%%%%%%%%%%%%%%%%%%%%%%%%%%%%%%%%%%%%%%%%%%%%%%%%%%%
\section{Za konec}

\begin{frame}

\frametitle{Zanimivosti in odprta vprašanja}

Omenimo še možno izbojšavo prejšnjega protokola in drugačen primer uporabe prej omenjenih protokolov.

\begin{itemize}

\item Zmanjševanje števila rezov (?)

\item Obstoj  delitve {\em super brez zavisti}

\item Razdeljevanje opravil

\end{itemize}

\end{frame}

%%%%%%%%%%%%%%%%%%%%%%%%%%%%%%%%%%%%%%%%%%%%%%%%%%%%%%%%%%%%%%%%%%%%%


\end{document}