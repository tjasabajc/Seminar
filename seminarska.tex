\documentclass[a4paper,12pt]{article}

\usepackage[slovene]{babel}
\usepackage{amsfonts,amssymb,amsmath}
\usepackage[utf8]{inputenc}
\usepackage[T1]{fontenc}
\usepackage{lmodern}
\usepackage{graphicx}
\usepackage{enumitem}
\usepackage{xcolor}


\def\N{\mathbb{N}} % mnozica naravnih stevil
\def\Z{\mathbb{Z}} % mnozica celih stevil
\def\Q{\mathbb{Q}} % mnozica racionalnih stevil
\def\R{\mathbb{R}} % mnozica realnih stevil
\def\C{\mathbb{C}} % mnozica kompleksnih stevil
\newcommand{\geslo}[2]{\noindent\textbf{#1} \quad \hangindent=1cm #2\\[-1pc]}

\def\qed{$\hfill\Box$}   % konec dokaza
\def\qedm{\qquad\Box}   % konec dokaza v matematičnem načinu
\newtheorem{izrek}{Izrek}
\newtheorem{trditev}{Trditev}
\newtheorem{posledica}{Posledica}
\newtheorem{lema}{Lema}
\newtheorem{pripomba}{Pripomba}
\newtheorem{definicija}{Definicija}
\newtheorem{zgled}{Zgled}

\newlist{protokol}{enumerate}{1}
\setlist[protokol, 1]{label =  \texttt{\textsf{ Korak \arabic*}}}
\setlist[protokol]{noitemsep}

\title{Delitev brez zavisti \\ 
\Large Seminar}
\author{Tjaša Bajc \\
Fakulteta za matematiko in fiziko \\
Oddelek za matematiko}
\date{24.\ februar 2017}

\begin{document}


%%%%%%%%%%%%%%%%%%%%%%%%%%%%%%%%%%%%%%%%%%%%%%%%%%%%%%%%%%%%%%%%%%%%%


\maketitle


%%%%%%%%%%%%%%%%%%%%%%%%%%%%%%%%%%%%%%%%%%%%%%%%%%%%%%%%%%%%%%%%%%%%%

\section{Uvod}

V vsakdanjem življenju se pogosto srečamo s problemom delitve nekega objekta ali množice objektov med $n$ oseb. Če želimo, da bo delitev pravična, se moramo najprej dogovoriti, kaj pravičnost pomeni, potem pa poiskati način, kako tako delitev doseči. Tradicionalno objekt, ki ga delimo, imenujemo torta, osebe, ki si torto delijo, pa igralci. %Ali naj dam zraven angleške besede?
Morda na prvi pogled ni jasno,zakaj bi se s tem vprašanjem sploh ukvarjali. Na primer, vsakdo zna torto razdeliti na $n$ delov in vsakemu igralcu dati kos, ki ustreza $1/n$ celote. 

Problem se pojavi, ker ljudje iste stvari različno vrednotimo, kar je seveda treba upoštevati pri razdeljevanju, če naj bo končna delitev pravična. Če je $A$ del celote, bomo to, koliko je kos $A$ vreden za igralca 1, označili z $\mu_1(A)$. Pri tem veljajo privzetki, da je za vse igralce mera celote $C$ enaka 1; %, $\mu_i(C) = 1, i \in \{1, 2, \ldots, n\}$
da kosov z enako mero med seboj ne razlikujemo; da je mogoče vsak kos nadalje razdeliti na $k$ enako velikih kosov, ter da je za vsaka kosa $A$ in $B$ mogoče kos $A$ obrezati tako, da bo $\mu(A) = \mu(B)$, ali obratno.


Skozi čas so se oblikovali različno strogi kriteriji, ki naj jim iskana delitev zadošča. Oglejmo si tri izmed njih.

\begin{definicija}
Naj bo $C$ množica, $\mu_1, \mu_2, \ldots, \mu_n$ po vrsti mere igralcev $1, 2, \ldots, n$ in $\{P_1, P_2, \ldots, P_n\}$ delitev celote $C$ na $n$ delov, kjer igralcu $i$ pripada kos $P_i$. Delitev je:

\begin{itemize}[noitemsep]
\item {\em sorazmerna\/}, če za vsak $i$ velja $\mu_i(P_i) \geq 1/n$;
\item {\em brez zavisti\/}, če za vsaka $i, j$ velja $\mu_i(P_i) \geq \mu_i(P_j)$;
\item{\em super brez zavisti\/}, če za vsaka $i, j$ velja $\mu_i(P_i) > 1/n$ in hkrati $\mu_i(P_j) < 1/n$ .								
\end{itemize}

\end{definicija}

\section{Pristopi k problemu pravične delitve}

\subsection{Metoda ``Reži in izberi''}

Za problem s samo dvema igralcema je zelo znana t.i. metoda ``Reži in izberi'', na katero se bomo pogosto sklicevali. Zaporedje korakov (kar bomo kasneje imenovali protokol) je v tem primeru zelo preprosto. Prvi igralec razdeli torto na dva dela (ki sta zanj enaka), drugi si nato izbere svoj kos, prvemu pa pripade preostanek. Ta delitev je {\em sorazmerna\/} in hkrati {\em brez zavisti\/}. Prvi igralec je dobil natančno $1/2$, drugi pa kvečjemu več. Če namreč njuni meri sovpadata, $\mu_1 = \mu_2$, je prav tako dobil natančno $1/2$, v nasprotnem primeru pa si lahko izbere večjega od obeh kosov.

Omenjeni primer $n = 2$ je edini, kjer pojma {\em proporcionalne delitve\/} in {\em delitve brez zavisti\/} sovpadata. Očitno je, da je {\em delitev brez zavisti\/} vedno {\em sorazmerna\/}, obratna implikacija pa ne drži.

\subsection{*** Algoritmi in rešitve s premikajočim se nožem}

\texttt{Besedilo moram še napisati}

\subsection{*** Izreki o obstoju ?? ?? ?? ?? ??}

\texttt{Besedilo moram še napisati}

\subsection{Protokoli}

Za nas bo najpomembnejši sistem pravil, ki ga bomo imenovali ``protokol'' (Sklic Sklic Sklic Sklic Sklic Sklic Sklic). Protokol je zaporedje navodil, ki ob upoštevanju vnaprej določenih pravil pripelje do razdelitve celote $C$ med $n$ igralcev. Od igralcev protokol zahteva dejanja, ki vplivajo na nadaljne korake, npr. ``Izberi $k$ kosov od preostalih $m$,'' ali ``Obreži enega od kosov,'' in podobno. Protokol od igralcev ne zahteva, da razkrijejo svoj načinu vrednotenja (mero). Pomembna značilnost protokola je, da nas do pravične razdelitve nujno pripelje v končno mnogo korakih, z drugimi besedami, da se ob upoštevanju pravil ne more zgoditi, da se postopek ne bi končal.

Zaporedje takih dejanj, ki so skladna s pravili protokola, in ki naj jih igralec povrsti izvede, ko je na potezi, bomo imenovali ``strategija''. Protokol je {\em sorazmeren}, če ima vsak od $n$ igralcev strategijo, ki mu bo zagotovila končni delež vsaj $1/n$ celote (v njegovem načinu vrednotenja), pri čemer mora biti ta rezultat neodvisen od strategij ostalih igralcev. Če ima vsak od igralcev strategijo, ki mu bo zagotovila ali kos, ki je zanj vreden največ, ali pa kos, ki je eden od nekaj enako velikih (največjih) kosov, je tak protokol {\em brez zavisti}. V tem primeru noben igralec svojega kosa ne bi menjal z drugim, kar je ekvivalentno temu, da nihče ni drugemu zavisten.

Dokaz, da obstaja npr. {\em sorazmeren protokol}, je sestavljen iz pravil, strategij za posamezne igralce in utemeljitve, zakaj predstaveljene strategije pripeljejo igralce do zahtevanih deležev (v tem primeru vsaj $1/n$). Pravila od strategij loči to, da njihovo izvajanje lahko nadzoruje neka tretja oseba, ki jo imenujmo sodnik. Dva odstavka nazaj smo zapisali, da igralcem nikoli ni potrebno razkriti, kako vrednotijo posamezne kose, torej sodnik o tem ne ve ničesar. Zato je recimo trditev ``Igralec 1 razdeli torto na 3 enake dele,'' strategija, ne pa pravilo. Pravilo bi se glasilo ``Igralec 1 razdeli torto na 3 dele.'' Sodnik ne more preveriti prve trditve, ker ne ve, kako igralec vrednoti kose. Druga trditev je preverljiva brez informacije o načinu vrednotenja, zato je pravilo.

Pri načinu zapisovanja protokolov bomo zaradi preglednosti prevzeli način, uporabljen v članku (Sklic Sklic Sklic Sklic Sklic Sklic). To pomeni, da bomo protokole predstavili kot zaporedje oštevilčenih korakov. Dokaz, da protokol res pripelje do pravične razdelitve, bomo zapisovali ``sproti'', torej med posamezniki koraki. Strategije bomo ločili od pravil tako, da jih bomo zapisali v oklepajih in v sivi barvi. S tem želimo doseči jasen zapis in omogočiti branje navodil ločeno od strategij.

V nadaljevanju so podrobno predstavljeni štirje protokoli, dva za {\em sorazmerno delitev} (za primer treh igralcev in v splošnem) in dva za {\em delitev brez zavisti} (ponovno najprej za primer treh igralcev in nato splošna rešitev).

% Ali sledi tudi dokaz, da obstaja delitev super brez zavisti ?  ?  ?  ?  ?  ?

\section{Sorazmerna delitev torte med tri igralce}

Naslednji protokol vodi do take razdelitve torte med tri ljudi, da vsak izmed njih svoj kos vrednoti kot vsaj eno tretjino celotne torte, torej je  {\em delitev sorazmerna}.


\begin{protokol}

\item Prvi igralec razdeli torto na tri dele \textcolor{gray}{(ki so zanj enako veliki)}.

\item Drugi igralec lahko ne naredi ničesar \textcolor{gray}{(če vsaj dva od treh kosov vrednoti kot večja od $1/3$ celote)}

ali pa dva kosa označi kot ``slaba''. \textcolor{gray}{(To stori, če sta vsak posebej zanj vredna strogo manj kot $1/3$ celote.)}

\item Če drugi igralec v prejšnjem koraku ni storil nič, smo končali in si igralci izberejo kose torte.

\textsl{Vrstni red izbiranja: Tretji igralec, drugi igralec, prvi igralec.}

\item [\textsf{\em Komentar}] V tem primeru je vsakdo dobil kos v velikosti vsaj  $1/3$ celote (v njegovem načinu vrednotenja). Tretji igralec izbira prvi, zato bo gotovo lahko izbral tak kos. Drugi igralec je vsaj dva kosa vrednotil kot večja od  $1/3$, torej je vsaj eden od teh kosov še na voljo po tem, ko tretji vzame svoj kos. Prvi igralec je torto razdelil na enake dele, torej bo v vsakem primeru zadovoljen s kosom, ki mu še ostane.

\item Sicer (če je v 2. koraku drugi igralec dva kosa označil kot ``slaba'') ponovimo drugi korak, le da je zdaj na potezi tretji igralec. Oznake drugega igralca prekrijemo, tretji igralec jih ne upošteva.

\item Če v prejšnjem koraku tretji igralec ni storil nič, smo končali in si igralci izberejo kose torte.

\textsl{Vrstni red izbiranja: Drugi igralec, tretji igralec, prvi igralec.}

\item [\textsf{\em Komentar}] Razmislek je podoben prejšnjemu. Drugi igralec lahko izbere kos, ki ga ni označil kot ``slabega'', saj izbira prvi. Tako nujno dobi vsaj $1/3$ celote. Tretjemu igralcu sta se zdela vsaj dva kosa enako velika (in oba večja od $1/3$) in vsaj en tak kos mu je še vedno na voljo. Prvi igralec je zadovoljen s preostalim kosom, saj je zanj vreden natačno $1/3$.

\item Če je tretji igralec v četrtem koraku dva kosa označil kot ``slaba'', mora prvi igralec vzeti kos, ki sta ga tako drugi kot tretji igralec določila za ``slabega''.

\item [\textsf{\em Komentar}]Tak kos gotovo obstaja. Prepričamo se lahko na konkretnem primeru ali pa upoštevamo Dirichletovo načelo - če imamo tri kose in štiri oznake ``slabo'', potem je vsaj en kos moral dobiti dve taki oznaki. S tem je prvi igralec dobil natanko $1/3$ torte, preostala igralca pa mislita, da je dobil strogo manj kot $1/3$. Po njunem vretnotenju je preostanek torte skupaj vreden več kot $2/3$ celote.

\item Preostala kosa torte ponovno združimo. To novo torto si drugi in tretji igralec razdelita po metodi ``Reži in izberi''.

\item [\textsf{\em Komentar}] Ker smo v prejšnjem komentarju pokazali, da si bosta razdelila več kot $2/3$ celote, z upoštevanjem uvodoma omenjenaga dejstva, da metoda ``Reži in izberi'' pripelje do sorazmerne delitve, lahko sklepamo, da je končna razdelitev res  {\em sorazmerna}.

\end{protokol}

\section{Sorazmerna delitev torte med $n$ igralcev}

Vprašanje, ali je mogoče predhodni protokol posplošiti na primer $n$ igralcev, je bilo kmalu razrešeno. Odkrit je bil protokol, ki ga opisujemo spodaj, posebej pa omenimo, da se v njem pojavi ideja obrezovanja kosov. Uvodoma smo omenili, da je mogoče za vsaka kosa $A$ in $B$ enega od njiju obrezati tako, da sta po obrezovanju enako velika. Ker med enako velikimi kosi ne ločimo, je zamisel o obrezovanju zelo koristna in pogosto uporabljena. 

Videli bomo, da je očitno, da se bo protokol končal v končno mnogo korakih, saj se ostanki nekega kroga razdeljevanja, ki nastanejo zaradi obrezovanja kosov, takoj v naslednjem krogu porabijo, za zadnja igralca pa sledimo preizkušeni metodi ``Reži in izberi''. 

Če zelo poenostavimo, je ideja podobna kot v konkretnem primeru za $n = 3$ zgoraj, kjer drugi in tretji igralec zavrneta premajhne kose. Prvi od $n$ igralcev si odreže kos, s katerim bi bil zadovoljen. Če bi bil kateri od preostalih igralcev zadovoljen že z manj, kos zmanjša do velikosti $1/n$ skladno s svojim vrednotenjem. Če pa se mu začetni kos zdi premajhen, z njim ne naredi nič, ampak ga poda naprej. Ko si kos ogledajo vsi igralci (in ga morebiti zmanjšujejo), se bo nujno našel nekdo, ki se mu bo kos zdel primeren. Ta dobi kos, ostali pa ponavljajo postopek, dokler ni razdeljena celotna torta.

\begin{protokol}

\item Prvi igralec od celote odreže kos $P_1$ \textcolor{gray}{(ki je zanj vreden natančno $1/n$)}.

\item Igralec 2 lahko ne naredi ničesar \textcolor{gray}{(če je zanj kos $P_1$ vreden manj od $1/n$)}

ali pa kos $P_1$ obreže \textcolor{gray}{(tako, da bo obrezan kos vreden natančno $1/n$)}.

Kos preimenujemo v $P_2$ (neodvisno od tega, ali je bil obrezan). Tisto, kar je igralec 2 odrezal stran, shranimo na posebno mesto za ostanke.

\item Za igralce od tretjega do zadnjega po vrsti naredimo naslednje: Igralec $i$ vzame kos $P_{i-1}$ od svojega predhodnika in ravna enako kot drugi igralec v prejšnjem koraku (torej kos obreže ali pa ne). Kos preimenujemo v $P_i$ in podamo naprej, morebitne ostanke pa shranjujemo na enem mestu.

\item [\textsf{\em Komentar}] Sedaj vsak igralec $i$ misli, da je kos $P_i$ velikosti $1/n$ ali manjši. Prav tako velja, da je kos $P_i$ vedno večji ali enak $P_{i+1}$, zato vsi mislijo, da je kos $P_n$ vreden kvečjemu $1/n$.

\item Kos $P_n$ dobi zadnji igralec, ki je obrezal kos, ali pa prvi igralec, če ga ni nihče obrezal.

\item [\textsf{\em Komentar}] Že v prejšnjem komentarju smo pokazali, da bo igralec, ki mu je bil dodeljen kos $P_n$, v vsakem primeru mislil, da je dobil natančno $1/n$ celote, saj ga sam naredil tako velikega. % Je to razumljivo?

\item Vse obrezane koščke združimo z ostankom torte (iz prvega koraka), tako da dobimo novo torto, ki jo želimo sorazmerno razdeliti med $n - 1$ ljudi. Ponavljamo enak postopek, dokler nam ne ostaneta samo dva igralca.

\item [\textsf{\em Komentar}] Kdor dobi kos v drugem krogu razdeljevanja, dobi natančno $1/(n - 1)$ nove torte. On in vse drugi se strinjajo, da je ta nova torta velika vsaj $(n - 1)/n$, torej je ta igralec dobil vsaj $1/n$ začetne torte.

\item Za zadnja igralca sledimo metodi ``Reži in izberi''.

\item [\textsf{\em Komentar}] Kot prej - vsakdo vrednoti svoj kos vsaj kot $1/n$ celote.

\end{protokol}

\section{Delitev brez zavisti med tri igralce}

Pri {\em delitvi brez zavisti} želimo celoto razdeliti tako, da svojega kosa nihče ne bi zamenjal s kosom drugega igralca. Spomnimo se, da kosov z enako mero med seboj ne ločujemo. V spodnjem protokolu je poleg obrezovanja pomemben tudi pojem ``nepreklicne prednosti'' enega igralca pred drugim in zamisel o postopni delitvi. Našteti trije elementi protokola so med seboj tesno povezani. Preprosto rečeno, z obrezovanjem pridemo do tega, da del torte razdelimo {\em brez zavisti}, del pa ostane nerazdeljen (ta del setavljajo koščki, ki so nastali zaradi obrezovanja večjih kosov). Da lahko razdelimo še ta preostanek, potrebujemo ``nepreklicno prednost'' enega igralca pred drugim. Podrobneje bomo to idejo razložili spodaj v kometnarjih ob konkretnem primeru.

D

Nepreklicna prednost

\begin{protokol}

\item Prvi igralec razdeli torto na tri dele \textcolor{gray}{(ki so zanj enako veliki)}.

\item Drugi igralec lahko ne naredi ničesar \textcolor{gray}{(če vsaj dva od treh kosov enako velika in največja)}

ali pa obreže enega od kosov. \textcolor{gray}{(Obreže največjega, in sicer tako, da bo po obrezovanju enako velik kot drugi največji. Tako dobimo dva enako velika največja kosa.)}

Če je obrezal kos, odrezani del shranimo na mesto za ostanke in imenujemo $O$ (kot ostanek).

\item Igralci si izberejo kose torte, ostankov ne upoštevamo. \textcolor{gray}{(Vsak izbere največji kos ali enega od enako velikih največjih kosov.)}

\textsl{Vrstni red izbiranja: Tretji igralec, drugi igralec, prvi igralec.}

Če je drugi igralec obrezal kos, ga je dolžan izbrati (če ga ni pred njim izbral tretji igralec).

\item [\textsf{\em Komentar}] Tako smo {\em brez zavisti} razdelili del torte in dobili delitev ${\{X_1, X_2, X_3, O\}}$. Oglejmo si, zakaj nihče izmed igralcev ni zavisten drugemu. Tretji igralec izbira prvi, zato seveda lahko izbere ustrezen kos. Drugi igralec je poskrbel, da bosta vsaj dva kosa enako velika in največja, zato mu je vsaj en tak gotovo še na voljo. Prvi igralec je torto razrezal, obrezani kos pa je nujno izbral eden od prejšnjih igralcev, ker tako zahtevajo pravila. Zato je tudi on zadovoljen s kosom, ki mu ostane.

\item Če drugi igralec ni obrezal kosa, smo končali, saj ni ostanka in smo že dosegli {\em delitev brez zavisti}. V nasprotnem primeru pa sta si drugi in tretji igralec razdelila en obrezan in en neobrezan kos. Tistega, ki je dobil neobrezan kos, imenujmo ``rezalec'', drugega pa ``ne-rezalec''. ``Rezalec'' sedaj razdeli $O$ na tri dele.

\item [\textsf{\em Komentar}] Rekli bomo, da ima prvi igralec ``nepreklicno prednost'' pred ne-rezalcem', saj je slednji v delni razdelitvi dobil obrezan kos, ki je za prvega gotovo manjši od $1/3$. Zato prvi igralec ne-rezalcu ne bo zavidal ne glede na to, kako bo $O$ razdeljen mednje.

\item Igralci si izberejo kose, na katere je bil razdeljen ostanek. \textcolor{gray}{(Vsak izbere največji kos ali enega od enako velikih največjih kosov.)}

\textsl{Vrstni red izbiranja: Ne-rezalec, prvi igralec, rezalec.}

\item [\textsf{\em Komentar}] Razdelili smo celotno torto.Vemo že, da je delitev ${\{X_1, X_2, X_3\}}$ {\em brez zavisti}, zato razmislimo le o delitvi ostanka. Ne-rezalec ne zavida nikomur, saj izbira prvi. Prvi igralec ne zavida ne-rezalcu zaradi ``nepreklicne prednosti'', kar smo utemeljili v prejšnjem komentarju, prav tako pa ne zavida rezalcu, saj ta izbira zadnji. Rezalec ne zavida nikomur, saj je sam razdelil $O$ na (zanj) enake kose. Da dobimo iskano delitev ${\{P_1, P_2, P_3\}}$, preprosto za vsakega od igralcev združimo kosa $X_i$ in $O_i$, kjer so $O_i$ kosi, na katere smo razdelili ostanek $O$.

\end{protokol}

\section{Delitev brez zavisti med $n$ igralcev}

Pri {\em delitvi brez zavisti} želimo celoto razdeliti tako, da svojega kosa nihče ne bi zamenjal s kosom drugega igralca. Spomnimo se, da kosov z enako mero med seboj ne ločujemo. V spodnjem protokolu je poleg obrezovanja pomemben tudi pojem ``nepreklicne prednosti'' enega igralca pred drugim in zamisel o postopni delitvi. Našteti trije elementi protokola so med seboj tesno povezani. Preprosto rečeno, z obrezovanjem pridemo do tega, da del torte razdelimo {\em brez zavisti}, del pa ostane nerazdeljen (ta del setavljajo koščki, ki so nastali zaradi obrezovanja večjih kosov). Da lahko razdelimo še ta preostanek, potrebujemo ``nepreklicno prednost'' enega igralca pred drugim. Podrobneje bomo to idejo razložili spodaj v kometnarjih ob konkretnem primeru.

D

Nepreklicna prednost

\begin{protokol}

\item Prvi igralec
 
 \end{protokol}

Vrednosti multiplikativne funkcije $f$  so določene že z vrednostmi pri
potencah praštevil. Če namreč razcepimo 
\[
n\ =\ p_1^{k_1} p_2^{k_2} \cdots p_r^{k_r},
\]
kjer so $p_1, p_2, \ldots, p_r$ različna praštevila in $k_1, k_2, \ldots, k_r \in \N$, je
\[
f(n)\ =\ f\!\left(p_1^{k_1}\right) f\!\left(p_2^{k_2}\right) \cdots f\!\left(p_r^{k_r}\right).
\]

Pomembni multiplikativni funkciji sta {\em Eulerjeva funkcija\/} $\varphi(n)$ in {\em M\"obiusova funkcija\/} $\mu(n)$. Oglejmo si nekaj njunih lastnosti.

%%%%%%%%%%%%%%%%%%%%%%%%%%%%%%%%%%%%%%%%%%%%%%%%%%%%%%%%%%%%%%%%%%%%%


\section{Eulerjeva funkcija}


\begin{definicija}
Za vse $n \in \N$ s $\varphi(n)$ označimo število 
celih števil iz množice $\{1, 2, \ldots, n\}$, ki so tuja številu $n$.
Preslikavo $\varphi: \N \to \N$ imenujemo \em{Eulerjeva funkcija}.
\end{definicija}

\begin{zgled}
Tabela %\ref{fi} prikazuje izračun prvih šest vrednosti funkcije $\varphi(n)$. V $n$-ti vrstici so 
krepko natisnjena števila med $1$ in $n$, ki so tuja številu $n$. %Slika~\ref{fi100} pa grafično prikazuje prvih 100 vrednosti funkcije $\varphi(n)$.
\begin{table}[h]
\[
\begin{array}{clc}
 n & \{1,2,\ldots, n\}          & \varphi(n)       \\
 \hline
protokol'' (Sklic Sklic Sklic Sklic Sklic Sklic Sklic). \\ Protokol je zaporedje navodil, ki ob upoštevarazde & \{{\bf 1}\}                    &     1      \\
 2 & \{{\bf 1},2 \}                &     1      \\
 3 & \{{\bf 1,2},3 \}             &     2      \\
 4 & \{{\bf 1},2,{\bf 3},4 \} &     2      \\
 5 & \{{\bf 1,2,3,4},5 \}       &     4      \\
 6 & \{{\bf 1},2,3,4,{\bf 5},6 \} &     2
\end{array}
\] 
\caption{Vrednosti funkcije $\varphi(n)$ za $n = 1,2,\ldots,6$}\label{fi}
\end{table}

%\begin{figure}[h]
%\includegraphics{fi100.pdf}
%\caption{Vrednosti funkcije $\varphi(n)$ za $n = 1,2,\ldots,100$}\label{fi100}
%\end{figure}
\end{zgled}

Računanje $\varphi(n)$ po definiciji je pri velikem $n$ zelo zamudno. Vendar ima 
Eulerjeva funkcija lepe lastnosti, zaradi katerih lahko njeno vrednost izračunamo
tudi pri velikem argumentu, če ga le znamo razcepiti na prafaktorje.

Če je $p$ praštevilo, med števili $1,2,\ldots, p$ edinole število $p$ ni tuje številu $p$,
torej je $\varphi(p) = p-1$. Skoraj prav tako preprosto lahko poiščemo vrednost $\varphi(n)$,
če je $n$ potenca nekega praštevila.

\begin{trditev}
\label{fipp}
Naj bo $p$ praštevilo in $k \in \N$. Potem je $\varphi(p^k) = p^k-p^{k-1}$.
\end{trditev}

\noindent
{\em Dokaz:\/}% ? tevilo $a$ je tuje številu $p^k$ natanko tedaj, ko ni večkratnik praštevila $p$.
Med števili $1,2,\ldots, p^k$ je natanko $p^k/p = p^{k-1}$ večkratnikov števila $p$, torej je $\varphi(p^k) = p^k-p^{k-1}$. 
\qed

\begin{izrek}
\label{fimult}
Eulerjeva funkcija je multiplikativna.
\end{izrek}

\noindent
{\em Dokaz:\/} Vzemimo tuji naravni števili $a$ in $b$. Zapišimo vsa števila med $1$ in $ab$
v obliki tabele z $a$ vrsticami in $b$ stolpci:
\[
\begin{array}{cccc}
1 & 2 & \cdots & \ b \\
b+1 & b+2 & \cdots & \ 2b \\
2b+1 & 2b+2 & \cdots & \ 3b \\
\vdots & \vdots & \cdots & \ \vdots \\
(a-1)b+1 & (a-1)b+2 & \cdots & \ ab 
\end{array}
\]
Za vsako število velja, da je tuje številu $ab$ natanko tedaj, ko je tuje številu $a$ in tuje številu $b$.
Vrednost $\varphi(ab)$ lahko torej dobimo tako, da preštejemo, koliko je v gornji tabeli števil, ki so tuja
tako številu $a$ kot tudi številu $b$. 

? tevila v posameznem stolpcu dajejo vsa isti ostanek pri deljenju z $b$. Torej so bodisi vsa tuja številu
$b$ bodisi mu ni tuje nobeno od njih. Stolpcev, katerih elementi so tuji številu $b$, je toliko, kot je
takih števil v prvi vrstici tabele, teh pa je ravno $\varphi(b)$.

Različna števila v posameznem stolpcu dajo različne ostanke pri deljenju z $a$. Če namreč števili
$k_1 b + r$ in $k_2 b + r$, kjer je $0 \le k_1, k_2 \le a-1$, dasta isti ostanek pri deljenju z $a$, je njuna razlika $(k_1 - k_2) b$
deljiva z $a$. Ker sta števili $a$ in $b$ tuji, sledi, da je z $a$ deljiva razlika $k_1 - k_2$.
To pa je možno le, če je $k_1 = k_2$, saj je $-(a-1) \le k_1 - k_2 \le a-1$. Ker je dolžina stolpca
enaka $a$, dobimo pri deljenju elementov stolpca z $a$ ravno vse možne ostanke $0, 1,\ldots, a-1$.
Torej je v vsakem stolpcu $\varphi(a)$ števil tujih $a$. 

To velja tudi za $\varphi(b)$ stolpcev, katerih elementi so tuji številu $b$. Potemtakem je v gornji tabeli
$\varphi(b)\varphi(a)$ števil, ki so tuja tako številu $b$ kot tudi številu $a$. 
Torej je $\varphi(ab) = \varphi(a)\varphi(b)$, kar pomeni, da je Eulerjeva funkcija multiplikativna. \qed


\begin{zgled}
Izračunajmo $\varphi(10^k)$. Ker je $10^k = 2^k 5^k$, je po izreku \ref{fimult} in trditvi \ref{fipp}
\[
\varphi(10^k)\ =\ \varphi(2^k)\varphi(5^k)\ =\ (2^k - 2^{k-1})(5^k - 5^{k-1})\ =\ 4\times 10^{k-1}.
\]
\end{zgled}


\begin{posledica}
\[
\varphi(n)\ =\ n \times \prod_{p\,|\,n} \left(1 - \frac{1}{p}\right),
\]
\end{posledica}
kjer $p$ preteče vse različne prafaktorje števila $n$.

\noindent
{\em Dokaz:\/} Naj bo $n = \prod_{i=1}^r p_i^{k_i}$,
kjer so $p_1, p_2, \ldots, p_r$ različna praštevila in $k_1, k_2, \ldots, k_r \in \N$. Po izreku 
\ref{fimult} in trditvi \ref{fipp} je potem
\begin{eqnarray*}
\varphi(n) &=& \prod_{i=1}^r \varphi\left(p_i^{k_i}\right)
\ =\ \prod_{i=1}^r \left(p_i^{k_i} - p_i^{k_i-1}\right) \\
 &=& \left(\prod_{i=1}^r p_i^{k_i}\right) \times \prod_{i=1}^r \left(1 - \frac{1}{p_i}\right)
\ =\ n \times \prod_{p\,|\,n} \left(1 - \frac{1}{p}\right). \qedm
\end{eqnarray*}


\begin{trditev}
Za vse $n \in \N$ velja enačba
\begin{equation}
\label{fisum}
\sum_{d\,|\,n} \varphi(d)\ =\  n,
\end{equation}
kjer $d$ preteče vse pozitivne delitelje števila $n$.
\end{trditev}

\noindent
{\em Dokaz:\/} Za vse delitelje $d$ števila $n$ označimo
\[
A_d\ =\ \left\{\frac{k n}{d};\ k \in \Z,\ 0 \le k < d,\ D(k,d) = 1\right\}.
\]
Recimo, da je $k_1 n/d_1 = k_2 n/d_2$, kjer je $D(k_1,d_1) = D(k_2,d_2) = 1$. Potem 
je $k_1 d_2 = k_2 d_1$, od koder sledi, da $d_1$ deli $d_2$ in obratno, kar pomeni,
da je $d_1 = d_2$. Od tod zaključimo, da so si množice $A_d$ paroma tuje,
torej je
\[
\left|\bigcup_{d\,|\,n} A_d\right|\ =\ \sum_{d\,|\,n} |A_d|\ =\ \sum_{d\,|\,n} \varphi(d).
\]
Po drugi strani pa je
\[
\bigcup_{d\,|\,n} A_d\ =\ \{0, 1, \ldots, n-1\}.
\]
Res, naj bo $k n/d \in A_d$. Ker $d$ deli $n$, je število $k n/d$ celo, iz $0 \le k < d$ pa sledi $0 \le k n/d < n$,
torej $k n/d \in \{0, 1, \ldots, n-1\}$. Vzemimo zdaj še poljuben $j \in \{0, 1, \ldots, n-1\}$
in označimo: $k = j/D(n,j)$, $d = n/D(n,j)$. Potem je $j = k D(n,j) = k n/d \in A_d$.

To pa pomeni, da je $\left|\bigcup_{d\,|\,n} A_d\right| = n$ in izrek je dokazan. \qed

\begin{izrek}[Eulerjev izrek]
Naj bosta $n \in \N$ in $a \in \Z$ tuji števili. Potem je
\[
a^{\varphi(n)} \equiv 1 \pmod{n}.
\]
\end{izrek}

\noindent
{\em Dokaz:\/} Naj bodo $k_1, k_2, \ldots, k_{\varphi(n)}$ vsa števila med $1$ in $n$, ki so tuja $n$. Če za indeksa $i,j \in \{1,2,\ldots,\varphi(n)\}$ velja $k_i a \equiv k_j a \!\!\!\pmod{n}$, sledi $n | (k_i a - k_j a)$ in zato $n | (k_i - k_j)$, saj sta števili $n$ in $a$ tuji. To pa je mogoče le, če je $i = j$. ? tevila  $k_1 a, k_2 a, \ldots, k_{\varphi(n)} a$ so torej med seboj paroma nekongruentna po modulu $n$. Ker so tuja številu $n$, je množica njihovih ostankov pri deljenju z $n$ enaka množici $\{k_1, k_2, \ldots, k_{\varphi(n)}\}$. Zato je $k_1 a\cdot k_2 a\cdots k_{\varphi(n)} a \equiv k_1\cdot k_2 \cdots k_{\varphi(n)} \pmod{n}$, od tod pa po krajšanju s produktom $ k_1\cdot k_2 \cdots k_{\varphi(n)}$, ki je tuj številu $n$, dobimo $a^{\varphi(n)} \equiv 1 \!\!\!\pmod{n}$. \qed

\begin{posledica}[mali Fermatov izrek]
Naj bo $p$ praštevilo in $a \in \Z$ celo število, ki ni deljivo s $p$. Potem je
\[
a^{p-1} \equiv 1 \pmod{p}.
\]
\end{posledica}

%%%%%%%%%%%%%%%%%%%%%%%%%%%%%%%%%%%%%%%%%%%%%%%%%%%%%%%%%%%%%%%%%%%%%


\section{M\"obiusova funkcija}


\begin{definicija}
Za vse $n \in \N$ naj bo
\[
\mu(n)\ =\ \left\{
\begin{array}{cl}
0, & \mbox{če\ } n \mbox{\ deljiv s kvadratom praštevila,} \\
(-1)^r, & \mbox{sicer,}
\end{array}
\right.
\]
kjer je $r$ število različnih prafaktorjev števila $n$.
Preslikavo $\mu: \N \to \Z$ imenujemo \em{M\"obiusova funkcija}.
\end{definicija}

\begin{zgled}
Tabela \ref{mi} prikazuje prvih nekaj vrednosti funkcije $\mu(n)$. 
\begin{table}[h]
\[
\begin{array}{c|*{10}{r}}
   n   & 1 & 2 & 3 & 4 & 5 & 6 & 7 & 8 & 9 & 10 \\
\hline
\mu(n) & \ \ 1 & -1 & -1 & \ \ 0 & -1 & \ \ 1 & -1 & \ \ 0 & \ \ 0 & \ \ 1
\end{array}
\]
\caption{Vrednosti funkcije $\mu(n)$}\label{mi}
\end{table}
\end{zgled}


\begin{izrek}
\label{mimult}
M\"obiusova funkcija je multiplikativna.
\end{izrek}

\noindent
{\em Dokaz:\/} Vzemimo tuji naravni števili $a$ in $b$. Če je število $ab$ deljivo s kvadratom praštevila,
velja to tudi za $a$ ali za $b$. V tem primeru je torej $\mu(ab) = 0 = \mu(a)\mu(b)$.
Če pa število $ab$ ni deljivo s kvadratom praštevila, velja to tudi za $a$ in za $b$. Naj bo $r$ število različnih
prafaktorjev števila $a$, $s$ pa število različnih prafaktorjev števila $b$. Potem je število različnih 
prafaktorjev števila $ab$ enako $r+s$, torej je v tem primeru $\mu(ab) = (-1)^{r+s} = (-1)^r (-1)^s = \mu(a)\mu(b)$.
\qed


\begin{trditev}
Za vse $n \in \N$ velja enačba
\begin{equation}
\label{mirek}
\sum_{d\,|\,n} \mu(d)\ =\ \left\{
\begin{array}{ll}
1, & n = 1, \\
0, & n > 1,
\end{array}
\right.
\end{equation}
kjer $d$ preteče vse pozitivne delitelje števila $n$.
\end{trditev}

\noindent
{\em Dokaz:\/} 
Zadošča seštevati po tistih deliteljih $d$ števila $n$, ki imajo same različne prafaktorje (sicer je
$\mu(d) = 0$). Imenujmo takšne delitelje {\em enostavni}. Naj bo $r$ število različnih prafaktorjev
števila $n$. ? tevilo enostavnih deliteljev števila $n$, ki imajo natanko $k$ prafaktorjev, je potem ${r \choose k}$,
prispevek takega delitelja h gornji vsoti pa znaša $\mu(d) = (-1)^k$. Torej je
\[
\sum_{d\,|\,n} \mu(d)\ =\ 
\sum_{k=0}^r (-1)^k {r \choose k} \ =\ 
\left\{
\begin{array}{ll}
1, & r = 0, \\
0, & r > 0
\end{array}
\right.\ =\ \left\{
\begin{array}{ll}
1, & n = 1, \\
0, & n > 1.
\end{array}
\right. \qedm
\]

\begin{pripomba}
Enačbo (\ref{mirek}) bi lahko uporabili tudi za (rekurzivno) definicijo funkcije $\mu(n)$:
\[
\mu(n)\ =\ \left\{
\begin{array}{cl}
1, & n = 1, \\
-\displaystyle\sum_{d\,|\,n, \,d < n} \mu(d), & n > 1.
\end{array}
\right.
\]
\end{pripomba}

%%%%%%%%%%%%%%%%%%%%%%%%%%%%%%%%%%%%%%%%%%%%%%%%%%%%%%%%%%%%%%%%%%%%%
M\"obiusova funkcija igra pomembno vlogo pri {\em M\"obiusovem obratu}, ki nam omogoča
izraziti aritmetično funkcijo $f(n)$, če poznamo funkcijo $g(n) = \sum_{d\,|\,n} f(d)$,
kjer $d$ preteče vse pozitivne delitelje števila $n$.

\begin{izrek} \em{(M\"obiusov obrat)} \ 
Za aritmetični funkciji $f, g$ velja:
\[
g(n)\ =\ \sum_{d\,|\,n} f(d)\quad \Longleftrightarrow\quad f(n)\ =\ \sum_{d\,|\,n} \mu\left(\frac{n}{d}\right)g(d)
\]
\end{izrek}

\noindent
{\em Dokaz:\/} 
Najprej vzemimo, da je $g(n) = \sum_{d\,|\,n} f(d)$ za vse $n \in \N$. Potem je
\begin{eqnarray*}
\sum_{d\,|\,n} \mu\left(\frac{n}{d}\right)g(d)
 &=& \sum_{d\,|\,n} \mu\left(\frac{n}{d}\right)\sum_{k\,|\,d} f(k)
\ =\ \sum_{k\,|\,n} f(k) \sum_{k\,|\,d\,|\,n} \mu\left(\frac{n}{d}\right) \\
 &=& \sum_{k\,|\,n} f(k) \sum_{a\,|\,(n/k)} \mu\left(a\right)\ =\ f(n).
 \end{eqnarray*}
Drugo enakost smo dobili z zamenjavo vrstnega reda seštevanja, tretjo z uvedbo nove spremenljivke $a = n/d$,
četrta pa sledi iz (\ref{mirek}).

Vzemimo zdaj, da je $f(n)\ =\ \sum_{d\,|\,n} \mu\left(\frac{n}{d}\right)g(d)$ za vse $n \in \N$. Potem je
\begin{eqnarray*}
\sum_{d\,|\,n} f(d)
 &=& \sum_{d\,|\,n} \sum_{k\,|\,d} \mu\left(\frac{d}{k}\right) g(k)
\ =\ \sum_{k\,|\,n} g(k) \sum_{k\,|\,d\,|\,n} \mu\left(\frac{d}{k}\right) \\
 &=& \sum_{k\,|\,n} g(k) \sum_{b\,|\,(n/k)} \mu\left(b\right)\ =\ g(n).
 \end{eqnarray*}
Drugo enakost smo dobili z zamenjavo vrstnega reda seštevanja, tretjo z uvedbo nove spremenljivke $b = d/k$,
četrta pa sledi iz (\ref{mirek}). \qed


\begin{zgled}
\label{tau}
\begin{itemize}
\item Iz enačbe (\ref{fisum}) sledi z M\"obiusovim obratom, da je 
\[
\varphi(n)\ =\ \sum_{d\,|\,n}\mu\left(\frac{n}{d}\right)d.
\]
\item Za vse $n \in \N$ s $\tau(n)$ označimo število vseh pozitivnih deliteljev števila $n$.
Torej je $\tau(n) = \sum_{d\,|\,n} 1$, od koder sledi z M\"obiusovim obratom, da je 
\[
\sum_{d\,|\,n}\mu\left(\frac{n}{d}\right)\tau(d)\ =\ 1.
\]
\item Za vse $n \in \N$ s $\sigma(n)$ označimo vsoto vseh pozitivnih deliteljev števila $n$.
Torej je $\sigma(n) = \sum_{d\,|\,n} d$, od koder sledi z M\"obiusovim obratom, da je 
\[
\sum_{d\,|\,n}\mu\left(\frac{n}{d}\right)\sigma(d)\ =\ n.
\]
\end{itemize}
\end{zgled}






%%%%%%%%%%%%%%%%%%%%%%%%%%%%%%%%%%%%%%%%%%%%%%%%%%%%%%%%%%%%%%%%%%%%%


\section{Kolobar aritmetičnih funkcij}

\begin{definicija}
Za aritmetični funkciji $f, g: \N \to \C$ in za vse $n \in \N$ naj bo 
\[
(f * g)(n) \ =\ \sum_{d\,|\,n} f(d)g\left(\frac{n}{d}\right).
\]
Aritmetična funkcija $f * g$ je {\em Dirichletova konvolucija\/} funkcij $f$ in $g$.
\end{definicija}

\begin{trditev}
\label{kolo}
Naj bodo $f$, $g$, $h$ aritmetične funkcije. Potem velja:
\begin{itemize}
\item[\rm (i)] $f * g \ =\ g * f$,
\item[\rm (ii)] $(f * g) * h \ =\ f * (g * h)$,
\item[\rm (iii)] $f * (g + h) \ =\ f * g + f * h$.
\end{itemize}
\end{trditev}

\noindent
{\em Dokaz:\/} 
\begin{itemize}
\item[\rm (i)] 
Trditev sledi iz zapisa Dirichletove konvolucije v simetrični obliki
\begin{equation}
\label{dk}
(f * g)(n) \ =\ \sum_{d e = n} f(d)g(e),
\end{equation}
kjer seštevamo po vseh urejenih parih naravnih števil $(d,e)$, katerih produkt je enak $n$.

\item[\rm (ii)] 
Z uporabo enačbe (\ref{dk}) izračunamo
\begin{eqnarray*}
((f * g) * h)(n) &=& \sum_{d e = n} (f * g)(d)h(e)
\ =\ \sum_{d e = n} \left(\sum_{a b = d} f(a)g(b)\right)h(e) \\
 &=& \sum_{a b e = n} f(a)g(b)h(e)
 \ =\ \sum_{a c = n} f(a) \sum_{b e = c} g(b)h(e) \\
 &=& \sum_{a c = n} f(a) (g * h)(c)
 \ =\ (f * (g * h))(n).
\end{eqnarray*}
Četrto enakost smo dobili z uvedbo nove spremenljivke $c = b e$.


\item[\rm (iii)]
Z uporabo enačbe (\ref{dk}) izračunamo
\begin{eqnarray*}
(f * (g + h))(n) &=& \sum_{d e = n} f(d)(g+h)(e)
\ =\ \sum_{d e = n} f(d)(g(e)+h(e)) \\
 &=& \sum_{d e = n} f(d)g(e) + \sum_{d e = n} f(d)h(e) \\
 &=& (f * g + f * h)(n). \qedm
\end{eqnarray*}

\end{itemize}

Iz trditve \ref{kolo} sledi, da je množica vseh aritmetičnih funkcij $f: \N \to \C$ z operacijama $+$ in $*$ komutativen kolobar. Imenujemo ga {\em Dirichletov kolobar} in označimo z $\cal D$. 

Funkcija $\varepsilon \in {\cal D}$, ki za vse $n \in \N$ zadošča enačbi
\[
\varepsilon(n)\ =\ \left\{
\begin{array}{ll}
1, & n = 1, \\
0, & n > 1,
\end{array}
\right.
\]
je enica kolobarja $\cal D$, saj za vse $f \in {\cal D}$ in $n \in \N$ velja
\[
(f * \varepsilon)(n) \ =\ \sum_{d e = n} f(d)\varepsilon(e) \ =\ f(n)\varepsilon(1) \ =\ f(n).
\]
Brez težav se lahko prepričamo tudi, da je $\cal D$ cel kolobar in da je funkcija $f \in {\cal D}$
obrnljiva natanko tedaj, ko $f(1) \ne 0$.

Zdaj lahko enačbo (\ref{mirek}) prepišemo v obliki
\[
\mu * \mathbf{1}\ = \ \varepsilon,
\]
kjer $\mathbf{1}$ označuje konstantno funkcijo z vrednostjo $1$. Z drugimi besedami, M\"obiusova funkcija
je inverz konstantne funkcije $\mathbf{1}$ glede na Dirichletovo konvolucijo:
\[
\mu \ =\  \mathbf{1}^{-1}.
\]
M\"obiusov obrat lahko torej zapišemo v obliki
\[
g\ =\ f * \mathbf{1} \quad \Longleftrightarrow \quad f\ =\ g * \mu,
\]
kjer njegova veljavnost postane očitna. Zgled \ref{tau} pa lahko prepišemo v obliki
\begin{eqnarray*}
\varphi * \mathbf{1} \ =\ {\rm id}_\N &\Longrightarrow& \varphi\ =\ \mu * {\rm id}_\N, \\
\tau \ =\ \mathbf{1} * \mathbf{1} &\Longrightarrow& \mu * \tau \ =\ \mathbf{1}, \\
\sigma \ =\ {\rm id}_\N * \mathbf{1} &\Longrightarrow& \mu * \sigma \ =\ {\rm id}_\N.
\end{eqnarray*}


\section*{Angleško-slovenski slovar strokovnih izrazov}


\geslo{arithmetic function}{aritmetična funkcija}

\geslo{coprime}{tuj}

\geslo{Dirichlet convolution}{Dirichletova konvolucija}

\geslo{Dirichlet ring}{Dirichletov kolobar, kolobar aritmetičnih funkcij}

\geslo{divisor}{delitelj}

\geslo{Euler's phi function, Euler's totient function}{Eulerjeva funkcija $\varphi$}

\geslo{Euler's theorem}{Eulerjev izrek}

\geslo{Fermat's little theorem}{mali Fermatov izrek}

\geslo{fundamental theorem of arithmetic}{osnovni izrek aritmetike}

\geslo{greatest common divisor}{največji skupni delitelj, največja skupna mera}

\geslo{least common multiple}{najmanjši skupni večkratnik}

\geslo{M\"obius function}{M\"obiusova funkcija $\mu$}

\geslo{M\"obius inversion}{M\"obiusov obrat, M\"obiusova inverzija}

\geslo{multiple}{večkratnik}

\geslo{prime}{praštevilo; praštevilski}

\geslo{prime factor}{prafaktor}

\geslo{prime number}{praštevilo}

\geslo{relatively prime}{tuj}




\begin{thebibliography}{1}
\bibitem{AiZ}
M.~Aigner in G.~M.~Ziegler, \emph{Proofs from THE BOOK}, 2.\ izdaja, Springer, Berlin--Heidelberg--New York, 2001.
\bibitem{CaW}
N.~Calkin in H.~S.~Wilf, Recounting the rationals,
\emph{Amer.~Math.~Monthly}  \textbf{107}  (2000),  360--363.
\bibitem{Gra}
J.~Grasselli, \emph{Elementarna teorija števil}, DMFA -- založništvo, Ljubljana, 2009.
\end{thebibliography}





\end{document}