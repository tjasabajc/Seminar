\documentclass[a4paper,12pt]{article}

\usepackage[slovene]{babel}
\usepackage{amsfonts,amssymb,amsmath}
\usepackage[utf8]{inputenc}
\usepackage[T1]{fontenc}
\usepackage{lmodern}
\usepackage{graphicx}
\usepackage{enumitem}
\usepackage{xcolor}


\def\N{\mathbb{N}} % mnozica naravnih stevil
\def\Z{\mathbb{Z}} % mnozica celih stevil
\def\Q{\mathbb{Q}} % mnozica racionalnih stevil
\def\R{\mathbb{R}} % mnozica realnih stevil
\def\C{\mathbb{C}} % mnozica kompleksnih stevil
\def\P{\mathbb{P}}
\newcommand{\geslo}[2]{\noindent\textbf{#1} \quad \hangindent=1cm #2\\[-1pc]}

\def\qed{$\hfill\Box$}   % konec dokaza
\def\qedm{\qquad\Box}   % konec dokaza v matematičnem načinu
\newtheorem{izrek}{Izrek}
\newtheorem{trditev}{Trditev}
\newtheorem{posledica}{Posledica}
\newtheorem{lema}{Lema}
\newtheorem{pripomba}{Pripomba}
\newtheorem{definicija}{Definicija}
\newtheorem{zgled}{Zgled}

\newlist{protokol}{enumerate}{1}
\setlist[protokol, 1]{label =  \textbf{ Korak \arabic*}}
\setlist[protokol]{noitemsep}

\title{Delitev brez zavisti \\ 
\Large Seminar}
\author{Tjaša Bajc \\
Fakulteta za matematiko in fiziko \\
Oddelek za matematiko}
\date{19.\ maj 2017}

\begin{document}


%%%%%%%%%%%%%%%%%%%%%%%%%%%%%%%%%%%%%%%%%%%%%%%%%%%%%%%%%%%%%%%%%%%%%


\maketitle


%%%%%%%%%%%%%%%%%%%%%%%%%%%%%%%%%%%%%%%%%%%%%%%%%%%%%%%%%%%%%%%%%%%%%

\section{Uvod}

V vsakdanjem življenju se pogosto srečamo s problemom delitve nekega objekta ali množice objektov med $n$ oseb. Če želimo, da bo delitev pravična, se moramo najprej dogovoriti, kaj pravičnost pomeni, potem pa poiskati način, kako tako delitev doseči. Tradicionalno objekt, ki ga delimo, imenujemo torta, osebe, ki si torto delijo, pa igralci. %Ali naj dam zraven angleške besede?
Morda na prvi pogled ni jasno, zakaj bi se s tem vprašanjem sploh ukvarjali. Na primer, vsakdo zna torto razdeliti na $n$ delov in vsakemu igralcu dati kos, ki ustreza $1/n$ celote. 

Problem se pojavi, ker ljudje iste stvari različno vrednotimo, kar je seveda treba upoštevati pri razdeljevanju, če naj bo končna delitev pravična. Če je $A$ del celote, bomo to, koliko je kos $A$ vreden za igralca 1, označili z $\mu_1(A)$. Pri tem veljajo privzetki, da je za vse igralce mera celote $C$ enaka 1; %, $\mu_i(C) = 1, i \in \{1, 2, \ldots, n\}$
da kosov z enako mero med seboj ne razlikujemo; da je mogoče vsak kos nadalje razdeliti na $k$ enako velikih kosov, ter da je za vsaka kosa $A$ in $B$ mogoče kos $A$ obrezati tako, da bo $\mu(A) = \mu(B)$, ali obratno.


Skozi čas so se oblikovali različno strogi kriteriji, ki naj jim iskana delitev zadošča. Oglejmo si tri izmed njih.

\begin{definicija}
Naj bo $C$ množica, $\mu_1, \mu_2, \ldots, \mu_n$ po vrsti mere igralcev $1, 2, \ldots, n$ in $\{P_1, P_2, \ldots, P_n\}$ delitev celote $C$ na $n$ delov, kjer igralcu $i$ pripada kos $P_i$. Delitev je:

\begin{itemize}[noitemsep]
\item {\em sorazmerna\/}, če za vsak $i$ velja $\mu_i(P_i) \geq 1/n$;
\item {\em brez zavisti\/}, če za vsaka $i, j$ velja $\mu_i(P_i) \geq \mu_i(P_j)$;
\item{\em super brez zavisti\/}, če za vsaka $i, j, i \neq j,$ velja $\mu_i(P_i) > 1/n$ in hkrati $\mu_i(P_j) < 1/n$ .								
\end{itemize}

\end{definicija}

\section{Pristopi k problemu pravične delitve}

\subsection{Metoda ``Reži in izberi''}

Za problem s samo dvema igralcema je zelo znana t.i. metoda ``Reži in izberi'', na katero se bomo pogosto sklicevali. Zaporedje korakov (kar bomo kasneje imenovali protokol) je v tem primeru zelo preprosto. Prvi igralec razdeli torto na dva dela (ki sta zanj enaka), drugi si nato izbere svoj kos, prvemu pa pripade preostanek. Ta delitev je {\em sorazmerna\/} in hkrati {\em brez zavisti\/}. Prvi igralec je dobil natančno $1/2$, drugi pa kvečjemu več. Če namreč njuni meri sovpadata, $\mu_1 = \mu_2$, je prav tako dobil natančno $1/2$, v nasprotnem primeru pa si lahko izbere večjega od obeh kosov.

Omenjeni primer $n = 2$ je edini, kjer pojma {\em sorazmerne delitve\/} in {\em delitve brez zavisti\/} sovpadata. Očitno je, da je {\em delitev brez zavisti\/} vedno {\em sorazmerna\/}, obratna implikacija pa ne drži.

\subsection{Protokoli}

Za nas bo najpomembnejši sistem pravil, ki ga bomo imenovali ``protokol''. Protokol je zaporedje navodil, ki ob upoštevanju vnaprej določenih pravil pripelje do razdelitve celote $C$ med $n$ igralcev. Od igralcev protokol zahteva dejanja, ki vplivajo na nadaljne korake, npr. ``Izberi $k$ kosov od preostalih $m$'' ali ``Obreži enega od kosov'' in podobno. Protokol od igralcev ne zahteva, da razkrijejo svoj način vrednotenja (mero). Pomembna značilnost protokola je, da nas do pravične razdelitve nujno pripelje v končno mnogo korakih, z drugimi besedami, da se ob upoštevanju pravil ne more zgoditi, da se postopek ne bi končal.

Zaporedje takih dejanj, ki so skladna s pravili protokola in ki naj jih igralec povrsti izvede, ko je na potezi, bomo imenovali ``strategija''. Protokol je {\em sorazmeren}, če ima vsak od $n$ igralcev strategijo, ki mu bo zagotovila končni delež vsaj $1/n$ celote (v njegovem načinu vrednotenja), pri čemer mora biti ta rezultat neodvisen od strategij ostalih igralcev. Če ima vsak od igralcev strategijo, ki mu bo zagotovila ali kos, ki je zanj vreden največ, ali pa kos, ki je eden od nekaj enako velikih (največjih) kosov, je tak protokol {\em brez zavisti}. V tem primeru noben igralec svojega kosa ne bi menjal z drugim, kar je ekvivalentno temu, da nihče ni drugemu zavisten.

Dokaz, da obstaja npr. {\em sorazmeren protokol}, je sestavljen iz pravil, strategij za posamezne igralce in utemeljitve, zakaj predstavljene strategije pripeljejo igralce do zahtevanih deležev (v tem primeru vsaj $1/n$). Pravila od strategij loči to, da njihovo izvajanje lahko nadzoruje neka tretja oseba, ki jo imenujmo sodnik. Dva odstavka nazaj smo zapisali, da igralcem nikoli ni potrebno razkriti, kako vrednotijo posamezne kose, torej sodnik o tem ne ve ničesar. Zato je recimo trditev ``Prvi igralec razdeli torto na 3 enake dele,'' strategija, ne pa pravilo. Pravilo bi se glasilo ``Prvi igralec razdeli torto na 3 dele.'' Sodnik ne more preveriti prve trditve, ker ne ve, kako igralec vrednoti kose. Druga trditev je preverljiva brez informacije o načinu vrednotenja, zato je pravilo.

Pri načinu zapisovanja protokolov bomo zaradi preglednosti prevzeli način, uporabljen v članku Bramsa in Taylorja \cite{bramstaylor}. To pomeni, da bomo protokole predstavili kot zaporedje oštevilčenih korakov. Dokaz, da protokol res pripelje do pravične razdelitve, bomo zapisovali ``sproti'', torej med posameznimi koraki. Strategije bomo ločili od pravil tako, da jih bomo zapisali v oklepajih in v sivi barvi. S tem želimo doseči jasen zapis in omogočiti branje navodil ločeno od strategij.

V nadaljevanju so podrobno predstavljeni štirje protokoli, dva za {\em sorazmerno delitev} (za primer treh igralcev in v splošnem) in dva za {\em delitev brez zavisti} (ponovno najprej za primer treh igralcev in nato splošna rešitev).

% Ali sledi tudi dokaz, da obstaja delitev super brez zavisti ?  ?  ?  ?  ?  ?

\section{Sorazmerna delitev torte med tri igralce}

Naslednji protokol vodi do take razdelitve torte med tri ljudi, da vsak izmed njih svoj kos vrednoti kot vsaj eno tretjino celotne torte, torej je {\em delitev sorazmerna}.


\begin{protokol}

\item Prvi igralec razdeli torto na tri dele \textcolor{gray}{(ki so zanj enako veliki)}.

\item Drugi igralec lahko ne naredi ničesar \textcolor{gray}{(če vsaj dva od treh kosov vrednoti kot večja ali enaka $1/3$ celote)} ali pa dva kosa označi kot ``slaba''. \textcolor{gray}{(To stori, če sta vsak posebej zanj vredna strogo manj kot $1/3$ celote.)}

\item Če drugi igralec v prejšnjem koraku ni storil nič, smo končali in si igralci izberejo kose torte.

\textsl{Vrstni red izbiranja: Tretji igralec, drugi igralec, prvi igralec.}

\item [\textbf { \em Komentar}] V tem primeru je vsakdo dobil kos v velikosti vsaj  $1/3$ celote (v svojem načinu vrednotenja). Tretji igralec izbira prvi, zato bo gotovo lahko izbral tak kos. Drugi igralec je vsaj dva kosa vrednotil kot večja od  $1/3$, torej je vsaj eden od teh kosov še na voljo po tem, ko tretji vzame svoj kos. Prvi igralec je torto razdelil na enake dele, torej bo v vsakem primeru zadovoljen s kosom, ki mu še ostane.

\item Sicer (če je v 2.~koraku drugi igralec dva kosa označil kot ``slaba'') ponovimo drugi korak, le da je zdaj na potezi tretji igralec. Oznake drugega igralca prekrijemo, tretji igralec jih ne upošteva.

\item Če v prejšnjem koraku tretji igralec ni storil nič, smo končali in si igralci izberejo kose torte.

\textsl{Vrstni red izbiranja: Drugi igralec, tretji igralec, prvi igralec.}

\item [\textbf{\em Komentar}] Razmislek je podoben prejšnjemu. Drugi igralec lahko izbere kos, ki ga ni označil kot ``slabega'', saj izbira prvi. Tako nujno dobi vsaj $1/3$ celote. Tretjemu igralcu sta se zdela vsaj dva kosa enako velika (in oba večja od $1/3$) in vsaj en tak kos mu je še vedno na voljo. Prvi igralec je zadovoljen s preostalim kosom, saj je zanj vreden natančno $1/3$.

\item Če je tretji igralec v četrtem koraku dva kosa označil kot ``slaba'', mora prvi igralec vzeti kos, ki sta ga tako drugi kot tretji igralec določila za ``slabega''.

\item [\textbf{\em Komentar}]Tak kos gotovo obstaja. Prepričamo se lahko na konkretnem primeru ali pa upoštevamo Dirichletovo načelo - če imamo tri kose in štiri oznake ``slabo'', potem je vsaj en kos moral dobiti dve taki oznaki. S tem je prvi igralec dobil natanko $1/3$ torte, preostala igralca pa mislita, da je dobil strogo manj kot $1/3$. Po njunem vretnotenju je preostanek torte skupaj vreden več kot $2/3$ celote.

\item Preostala kosa torte ponovno združimo. To novo torto si drugi in tretji igralec razdelita po metodi ``Reži in izberi''.

\item [\textbf{\em Komentar}] Ker smo v prejšnjem komentarju pokazali, da si bosta razdelila več kot $2/3$ celote, z upoštevanjem uvodoma omenjenega dejstva, da metoda ``Reži in izberi'' pripelje do sorazmerne delitve, lahko sklepamo, da je končna razdelitev res  {\em sorazmerna}.

\end{protokol}

\section{Sorazmerna delitev torte med $n$ igralcev}

Vprašanje, ali je mogoče predhodni protokol posplošiti na primer $n$ igralcev, je bilo kmalu razrešeno. Odkrit je bil protokol, ki ga opisujemo spodaj, posebej pa omenimo, da se v njem pojavi ideja obrezovanja kosov. V uvodu smo omenili, da je mogoče za vsaka kosa $A$ in $B$ enega od njiju obrezati tako, da sta po obrezovanju enako velika. Ker med enako velikimi kosi ne ločimo, je zamisel o obrezovanju zelo koristna in pogosto uporabljena. 

Videli bomo, da je očitno, da se bo protokol končal v končno mnogo korakih, saj se ostanki nekega kroga razdeljevanja, ki nastanejo zaradi obrezovanja kosov, takoj v naslednjem krogu porabijo, za zadnja igralca pa sledimo preizkušeni metodi ``Reži in izberi''. 

Če zelo poenostavimo, je ideja podobna kot v konkretnem primeru za $n = 3$ zgoraj, kjer drugi in tretji igralec zavrneta premajhne kose. Prvi od $n$ igralcev si odreže kos, s katerim bi bil zadovoljen. Če bi bil kateri od preostalih igralcev zadovoljen že z manj, kos zmanjša do velikosti $1/n$ skladno s svojim vrednotenjem. Če pa se mu začetni kos zdi premajhen, z njim ne naredi nič, ampak ga poda naprej. Ko si kos ogledajo vsi igralci (in ga morebiti zmanjšujejo), se bo nujno našel nekdo, ki se mu bo kos zdel primeren. Ta dobi kos, ostali pa ponavljajo postopek, dokler ni razdeljena celotna torta.

\begin{protokol}

\item Prvi igralec od celote odreže kos $P_1$ \textcolor{gray}{(ki je zanj vreden natančno $1/n$)}.

\item Drugi igralec lahko ne naredi ničesar \textcolor{gray}{(če je zanj kos $P_1$ vreden manj od $1/n$)} ali pa kos $P_1$ obreže \textcolor{gray}{(tako, da bo obrezan kos vreden natančno $1/n$)}.

Kos preimenujemo v $P_2$ (neodvisno od tega, ali je bil obrezan). Tisto, kar je drugi igralec odrezal stran, shranimo na posebno mesto za ostanke.

\item Za igralce od tretjega do zadnjega po vrsti naredimo naslednje: Igralec $i$ vzame kos $P_{i-1}$ od svojega predhodnika in ravna enako kot drugi igralec v prejšnjem koraku (torej kos obreže ali pa ne). Kos preimenujemo v $P_i$ in podamo naprej, morebitne ostanke pa shranjujemo na enem mestu.

\item [\textbf{\em Komentar}] Preden igralec kos poda naprej (torej po morebitnem obrezovanju), meni, da je kos velik $1/n$ ali manj. Prav tako velja, da se kos z vsakim novim igralcem kvečjemu zmanjša, zato vsi mislijo, da je na koncu kos $P_n$ vreden največ $1/n$.

\item Kos $P_n$ dobi zadnji igralec, ki je obrezal prejeti kos, ali pa prvi igralec, če ga ni nihče obrezal.

\item [\textbf{\em Komentar}] Že v prejšnjem komentarju smo pokazali, da bo igralec, ki mu je bil dodeljen kos $P_n$, v vsakem primeru mislil, da je dobil natančno $1/n$ celote, saj ga je sam naredil tako velikega.

\item Vse obrezane koščke združimo s preostankom začetne torte (iz prvega koraka) in tako dobimo novo torto, ki jo želimo sorazmerno razdeliti med $n - 1$ ljudi. Ponavljamo enak postopek, dokler nam ne ostaneta samo dva igralca.

\item [\textbf{\em Komentar}] Kdor dobi kos v drugem krogu razdeljevanja, dobi natančno $1/(n - 1)$ nove torte. On in vsi drugi se strinjajo, da je ta nova torta velika vsaj $(n - 1)/n$, torej je ta igralec dobil vsaj $1/n$ začetne torte.

\item Za zadnja igralca sledimo metodi ``Reži in izberi''.

\item [\textbf{\em Komentar}] Kot prej - vsakdo vrednoti svoj kos vsaj kot $1/n$ celote.

\end{protokol}

\section{Delitev brez zavisti med tri igralce}

Pri {\em delitvi brez zavisti} želimo celoto razdeliti tako, da svojega kosa nihče ne bi zamenjal s kosom drugega igralca. Spomnimo se, da kosov z enako mero med seboj ne ločujemo. V spodnjem protokolu je poleg obrezovanja pomemben tudi pojem ``nepreklicne prednosti'' enega igralca pred drugim in zamisel o postopni delitvi. Našteti trije elementi protokola so med seboj tesno povezani. Preprosto rečeno, z obrezovanjem pridemo do tega, da del torte razdelimo {\em brez zavisti}, del pa ostane nerazdeljen (ta del sestavljajo koščki, ki so nastali zaradi obrezovanja večjih kosov). Da lahko razdelimo še ta preostanek, potrebujemo ``nepreklicno prednost'' enega igralca pred drugim. Podrobneje bomo to idejo razložili spodaj v komentarjih ob konkretnem primeru.


\begin{protokol}

\item Prvi igralec razdeli torto na tri dele \textcolor{gray}{(ki so zanj enako veliki)}.

\item Drugi igralec lahko ne naredi ničesar \textcolor{gray}{(če vrednoti vsaj dva od treh kosov kot enako velika in največja)} ali pa obreže enega od kosov. \textcolor{gray}{(Obreže največjega, in sicer tako, da bo po obrezovanju enako velik kot drugi največji. Tako dobimo dva enako velika največja kosa.)}

Če je obrezal kos, odrezani del shranimo na mesto za ostanke in ga imenujemo $O$ (kot ostanek).

\item Igralci si izberejo kose torte, ostankov ne upoštevamo. \textcolor{gray}{(Vsak izbere največji kos ali enega od enako velikih največjih kosov.)} Zahtevamo, da drugi igralec vzame obrezani kos, če ga ni pred njim izbral tretji igralec.

\textsl{Vrstni red izbiranja: Tretji igralec, drugi igralec, prvi igralec.}

\item [\textbf{\em Komentar}] Tako smo {\em brez zavisti} razdelili del torte in dobili delitev ${\{X_1, X_2, X_3, O\}}$. Oglejmo si, zakaj nihče izmed igralcev ni zavisten drugemu. Tretji igralec izbira prvi, zato seveda lahko izbere ustrezen kos. Drugi igralec je poskrbel, da bosta vsaj dva kosa enako velika in največja, zato mu je vsaj en tak gotovo še na voljo. Prvi igralec je torto razrezal, obrezani kos pa je nujno izbral eden od prejšnjih igralcev, ker tako zahtevajo pravila. Zato je tudi on zadovoljen s kosom, ki mu ostane.

\item Če drugi igralec ni obrezal kosa, smo končali, saj ni ostanka in smo že dosegli {\em delitev brez zavisti}. V nasprotnem primeru pa sta si drugi in tretji igralec razdelila en obrezan in en neobrezan kos. Tistega, ki je dobil neobrezan kos, imenujmo ``rezalec'', drugega pa ``ne-rezalec''. ``Rezalec'' sedaj razdeli $O$ na tri dele.

\item [\textbf{\em Komentar}] Rekli bomo, da ima prvi igralec ``nepreklicno prednost'' pred ne-rezalcem, saj je slednji v delni razdelitvi dobil obrezan kos, ki je za prvega gotovo manjši od $1/3$. Zato prvi igralec ne-rezalcu ne bo zavidal ne glede na to, kako bo $O$ razdeljen mednje.

\item Igralci si izberejo kose, na katere je bil razdeljen ostanek. \textcolor{gray}{(Vsak izbere največji kos ali enega od enako velikih največjih kosov.)}

\textsl{Vrstni red izbiranja: Ne-rezalec, prvi igralec, rezalec.}

\item [\textbf{\em Komentar}] Razdelili smo celotno torto. Vemo že, da je delitev ${\{X_1, X_2, X_3\}}$ {\em brez zavisti}, zato razmislimo le o delitvi ostanka. Ne-rezalec ne zavida nikomur, saj izbira prvi. Prvi igralec ne zavida ne-rezalcu zaradi ``nepreklicne prednosti'', kar smo utemeljili v prejšnjem komentarju, prav tako pa ne zavida rezalcu, saj ta izbira zadnji. Rezalec ne zavida nikomur, saj je sam razdelil $O$ na (zanj) enake kose. Da dobimo iskano delitev ${\{P_1, P_2, P_3\}}$, preprosto za vsakega od igralcev združimo kosa $X_i$ in $O_i$, kjer so $O_i$ kosi, na katere smo razdelili ostanek $O$.

\end{protokol}

\section{Delitev brez zavisti med $n$ igralcev}

Zadnji protokol je najdaljši in najmanj intuitiven, saj je {\em delitev brez zavisti} težje doseči kot {\em sorazmerno}. Kot v prejšnjem protokolu bomo z obrezovanjem prišli do delne razdelitve.
Čeprav je protokol uporaben za splošen $n$, bomo zaradi jasnosti opisali primer za $n = 4$. 


\begin{protokol}

\item Drugi igralec razdeli torto na štiri dele \textcolor{gray}{(ki so zanj enako veliki)} in jih poljubno razdeli med vse štiri igralce.

\item Vsak od preostalih treh igralcev ima možnost ugovarjati. \textcolor{gray}{(To stori, če in samo če zavida kateremu od drugih igralcev)}

\item Če ni ugovorov, smo končali in vsakdo obdrži dodeljeni kos.

\item Sicer izberemo najmanjši tak $i$, da je $i$-ti igralec ugovarjal. Brez škode za splošnost privzemimo, da je ugovarjal prvi igralec. Njegov kos imenujmo $B$, neki drug kos imenujmo $A$. \textcolor{gray}{(Tako imenujemo kos tistega igralca, ki mu je prvi igralec zavidal.)} % Res ??? Ali je to kos 2. igralca?

\item [\textbf{\em Komentar}] Jasno je, da se prvemu igralcu kos $A$ zdi večji od $B$, sicer ne bi bil zavisten. Iz prvega koraka vemo, da se drugemu igralcu zdita kosa enako velika, saj je on razrezal torto. Na tem mestu preostala kosa (torej tista izmed začetnih štirih, ki nista ne $A$ in ne $B$) združimo in damo na stran, ta del bomo razdelili kasneje.

\item Prvi igralec določi naravno število $r > \max\{ \frac{6a}{a-b}, 6 \}$, kjer je $a = \mu_1(A)$ ter $b = \mu_1(B)$.

\item Drugi igralec razdeli tako $A$ kot $B$ na $r$ delov \textcolor{gray}{(ki so zanj enako veliki)}.

\item Prvi igralec izbere \textcolor{gray}{(najmanjše)} tri podkose kosa $B$ in jih imenuje $Z_1, Z_2, Z_3$ ter tri podkose kosa $A$, ki jih imenuje $Y_1, Y_2, Y_3$. Za podkose kosa $A$ ima dve možnosti. Ali vzame tri \textcolor{gray}{(največje)} podkose \textcolor{gray}{(če meni, da je vsak od teh treh posebej večji od največjega izmed $Z_1, Z_2, Z_3$)} in največ dva od teh treh obreže \textcolor{gray}{(na velikost najmanjšega od izbranih podkosov kosa $A$)} ali pa vzame \textcolor{gray}{(največji)} podkos in ga razdeli na tri \textcolor{gray}{(enako velike)} dele.

\item [\textbf{\em Komentar}] Strategija prvemu igralcu zagotavlja, da bo v vsakem primeru dobil tri enako velike kose (podkosi $Y_1, Y_2, Y_3$), ki so večji od največjega izmed $Z_1, Z_2, Z_3$. V prvem primeru je to očitno, v drugem pa razmislimo takole: Naj bodo $A_1, A_2, \ldots, A_r$ ter $B_1, B_2, \ldots, B_r$ kosi, na katere je drugi igralec v šestem koraku razdelil kosa $A$ in $B$. Za $i = 1, 2, \ldots, r$ označimo $a_i = \mu_1 (A_i)$ in  $b_i = \mu_1 (B_i)$. Brez škode za splošnost lahko vzamemo, da je $a_1 \geq a_2 \geq \ldots \geq a_r$ in $b_1 \geq b_2 \geq \ldots \geq b_r$. 
Potem je 
$\{ \mu_1(Z_1), \mu_1(Z_2), \mu_1(Z_3)\} = \{b_{r-2},b_{r-1},b_r\}$ in $\mu_1(Y_1) = \mu_1(Y_2) = \mu_1(Y_3) = \frac{a_1}{3}$. Nadalje velja $$b = \sum_{i=1}^r b_i \geq  \sum_{i=1}^{r-2} b_i \geq (r-2) b_{r-2},$$ torej $b_{r-2} \leq b/(r-2)$.

V drugem primeru pa velja $a_3 \leq b_{r-2}$. Da je kos $b_{r-2}$ strogo manjši od tretjine kosa $a_1$, bomo dokazali s protislovjem. Recimo, da je $a_1/3 \leq b_{r-2}$. Potem je
%$ a = \sum_{i=1}^r a_i 
% \leq 2 a_i + (r-2)a_3  
% \leq 6 b_{r-2} + (r-2) b_{r-2} 
% = (r +    4)b_{r-2} \leq \frac{r+4}{r-2} b$

\begin{align*}
 a &= \sum_{i=1}^r a_i \\
 &\leq 2 a_1 + (r-2)a_3  \\
 &\leq 6 b_{r-2} + (r-2) b_{r-2} \\
 &= (r +    4)b_{r-2} \\
 &\leq \frac{r+4}{r-2}b.
\end{align*}

Iz  $r >\frac{6a}{a-b}$ pa sledi $a < \frac{r}{r-6} b$, torej $\frac{r}{r-6} b < a \leq \frac{r+4}{r-2}$ in od tod $r(r-2) < (r+4)(r-6)$, kar se poenostavi v protislovje. Sklepamo, da je $a_1/3 > b_{r-2}$, s čimer je trditev o prvem igralcu dokazana.


%%%%%		PETKOVŠEK - preberi in ukrepaj

 Drugi igralec pa misli, da so $Z_1, Z_2, Z_3$ vsi enako veliki in vsak od teh posebej vsaj tako velik kot $Y_1, Y_2, Y_3$.

\item Tretji igralec vzame teh 6 posebej označenih kosov in ne stori nič \textcolor{gray}{(če meni, da sta vsaj dva kosa enako velika in največja)} ali pa obreže enega izmed šestih kosov. \textcolor{gray}{(Obreže največjega na velikost drugega največjega.)}

\item  Sedaj si štirje igralci izberejo vsak po enega izmed kosov $Y_1, Y_2, Y_3, Z_1, Z_2, Z_3$. \textcolor{gray}{(Vsak izbere največji kos ali enega od nekaj enako velikih največjih kosov)}. Dodatno zahtevamo naslednje: Tretji igralec mora izbrati kos, ki ga je obrezal v prejšnjem koraku, če mu je ta še na voljo. Drugi igralec mora vzeti enega od $Z$-jev, prvi pa enega od $Y$-ov.

\textsl{Vrstni red izbiranja: Četrti igralec, tretji igralec, drugi igralec, prvi igralec.}

\item [\textbf{\em Komentar}] Tako smo torto delno razdelili. Dobimo razdelitev ${\{X_1, X_2, X_3, X_4, O_1\}}$, kjer je ${\{X_1, X_2, X_3, X_4\}}$ {\em delitev brez zavisti}, $O_1$ pa ostanek, sestavljen iz vseh do sedaj nastalih obrezkov in zlepka neporabljenih kosov iz drugega koraka. Velja še, da se prvemu igralcu zdi kos $X_1$ strogo večji od kosa $X_2$, ki je pripadel drugemu igralcu. Označimo (pozitivno) razliko med kosoma $X_1$ in $X_2$ z $\epsilon$.

\item Prvi igralec imenuje pozitivno naravno število $s$. \textcolor{gray}{(Za $s$ naj velja, da je $[4\mu_1(O_1)/5]^s$ še zmeraj manjše od $\epsilon$}. 

\item [\textbf{\em Komentar}] Število $s$ pomeni število ponovitev osnovne metode ``Reži in izberi'', kot jo bomo opisali spodaj v korakih od 11.~do 14.

% MA NJ KA !

\item Prvi igralec razdeli $O_1$ na pet delov \textcolor{gray}{(ki so zanj enako veliki)}.

\item Drugi igralec vzame teh pet kosov, izbere \textcolor{gray}{(največje)} tri izmed njih in obreže največ dva od teh treh \textcolor{gray}{(na velikost najmanjšega tako, da bodo od dobljenih petih kosov obstajali vsaj trije enako veliki največji kosi)}. Obrezke vedno shranjujemo ločeno.

\item Tretji igralec dobi pet (morda obrezanih) kosov od drugega igralca, izbere \textcolor{gray}{(največja)} dva in enega od njiju morebiti obreže. \textcolor{gray}{(To stori tako, da bosta na koncu vsaj dva kosa od petih kosov enako velika in največja)}.

\item Zdaj (delno) razdelimo ostanek $O_1$ med štiri igralce tako, da vsak izbere enega od petih kosov \textcolor{gray}{(kot prej - vsak izbere največji kos ali enega od enako velikih največjih kosov)}. Ob tem dodatno zahtevamo, da drugi in tretji igralec vzameta kos, ki sta ga sama obrezala, če jima je tak kos še na voljo, ko prideta na vrsto za izbiranje.

\textsl{Vrstni red izbiranja: Četrti igralec, tretji igralec, drugi igralec, prvi igralec.}

\item Še $(s - 1)$-krat ponovimo korake od 11.~do 14, vsakokrat na ostanku predhodne ponovitve. Torej, vsakokrat razdelimo, kar je ostalo nerazdeljeno v 13.~koraku.

\item [\textbf{\em Komentar}]  S tem smo {\em brez zavisti} razdelili še en del torte, konkretno del ostanka $O_1$. Vsak igralec združi kos, ki ga je prejel v 9.~koraku, s koščki, ki jih je dobil z $s$ ponovitvami prej omenjenih korakov, in zlepek vseh svojih koščkov imenuje $X'_i$. Dobili smo torej razdelitev ${\{X'_1, X'_2, X'_3, X'_4, O_2\}}$ začetne torte, kjer je ${\{X'_1, X'_2, X'_3, X'_4\}}$ {\em delitev brez zavisti}. Dosegli smo še, da prvi igralec vrednoti svoj kos $X'_1$ kot večjega od unije kosov $X'_2$ in $O_2$. Zato lahko povemo, da ima prvi igralec ``nepreklicno prednost'' pred drugim igralcem ter par $(1,2)$ dodamo v (do sedaj prazno) množico $\P$. Množica $\P$ je podmnožica ${\{1, 2, 3, 4\}} \times {\{1, 2, 3, 4\}}$, ki vsebuje pare $(i, j)$, za katere velja, da ima $i$-ti igralec ``nepreklicno prednost'' pred $j$-tim igralcem (od tod ime množice - ``P'' kot prednost).

\item Drugi igralec razdeli $O_2$ na 12 kosov \textcolor{gray}{(ki so zanj enako veliki)}.

\item Vsak od igralcev se opredeli kot bodisi igralec tipa $E$ \textcolor{gray}{(če se mu zdijo vsi kosi enako veliki)} bodisi igralec tipa $R$ \textcolor{gray}{(če se mu zdita vsaj dva kosa različno velika)}. Drugi igralec je seveda tipa $E$.

\item Če velja, da je  $R \times E \subset \P$, damo vseh 12 kosov igralcem iz $E$ tako, da vsak dobi enako število kosov (vemo, da se jim zdijo kosi enako veliki). S tem smo končali delitev celotne torte.

\item V nasprotnem primeru pa izberemo leksikografsko najmanjši par $(i, j)$ iz $R \times E$, ki ni v $\P$. Sedaj se vrnemo v 4.~korak, kjer vlogo prvega igralca dobi $i$-ti igralec, vlogo drugega pa $j$-ti. Namesto celotne torte sedaj razdeljujemo $O_2$ (ponovno združimo 12 kosov iz 16.~koraka).

\item Ponavljamo korake od 5.~do 18.

 \item [\textbf{\em Komentar}] Vsakokrat v 15.~koraku v množico $\P$ dodamo nov par $(i, j)$. Ker velja $R \times E \subset {\{1, 2, 3, 4\}} \times {\{1, 2, 3, 4\}}$ in $\P \subset {\{1, 2, 3, 4\}} \times {\{1, 2, 3, 4\}}$, mora po največ 16 korakih veljati tudi $R \times E \subset \P$. Čim to velja, v 18.~koraku zaključimo postopek, saj smo razdelili vso torto.
 
 \end{protokol}
 
 \section{Zaključek}
 
 V seminarju smo predstavili nekaj osnovnih definicij pravične delitve in pojem protokola. Navedli smo dva protokola, ki vodita do sorazmerne delitve dane celote med igralce, kjer vsak od igralcev dobi najmanj $1/n$ glede na lastni način vrednotenja. V tem primeru ne moremo izključiti možnosti, da bi se nekemu igralcu ob končni razdelitvi zdelo, da je nekdo drug dobil {\em več} od njega in bi mu bil zato zavisten. Druga dva protokola poskrbita tudi za to, saj nas pripeljeta do take razdelitve, da nihče izmed igralcev svojega kosa ne bi želel zamenjati s kosom drugega igralca.
 
 Kot zanimivost povejmo še, da je mogoče vse štiri protokole uporabiti tudi za reševanje ``problema z opravili'', kjer želimo med igralce razdeliti neke naloge (delo). Tedaj si igralci prizadevajo za {\em čim manjši} delež celote, čemur smiselno prilagodimo prej naštete protokole.

%%%%%%%%%%%%%%%%%%%%%%%%%%%%%%%%%%%%%%%%%%%%%%%%%%%%%%%%%%%%%%%

\section*{Angleško-slovenski slovar strokovnih izrazov}


\geslo{cut-and-choose protocol}{metoda ``Reži in izberi''}

\geslo{measure}{mera}

\geslo{allocation}{razdelitev}

\geslo{partial allocation}{delna razdelitev}

\geslo{proportional}{sorazmeren}

\geslo{envy-free}{brez zavisti}

\geslo{trimming}{obrezovanje}

\geslo{irrevocable advantage}{nepreklicna prednost}

%%%%%%%%%%%%%%%%%%%%%%%%%%%%%%%%%%%%%%%%%%%%%%%%%%%%%%%%%%%%%%%

\begin{thebibliography}{5}
\bibitem{barbanel}
J.~B.~Barbanel: Super envy-free cake division and independence of measures, \emph{J. Math. Anal. Appl.} \textbf{197} (1996) 54 -- 50.
\bibitem{bramstaylor}
S.~J.~Brams, A.~D.~Taylor: An envy-free cake division protocol, \emph{Amer. Math. Monthly} \textbf{102} (1995) 9 -- 18.
\end{thebibliography}





\end{document}