\documentclass{beamer}

\usepackage[slovene]{babel}
\usepackage{amsfonts,amssymb}
\usepackage[utf8]{inputenc}
\usepackage{lmodern}
\usepackage[T1]{fontenc}

\usetheme{Madrid}
\usecolortheme{wolverine}

\def\N{\mathbb{N}} % mnozica naravnih stevil
\def\Z{\mathbb{Z}} % mnozica celih stevil
\def\Q{\mathbb{Q}} % mnozica racionalnih stevil
\def\R{\mathbb{R}} % mnozica realnih stevil
\def\C{\mathbb{C}} % mnozica kompleksnih stevil


\def\qed{$\hfill\Box$}   % konec dokaza
\newtheorem{izrek}{Izrek}
\newtheorem{trditev}{Trditev}
\newtheorem{posledica}{Posledica}
\newtheorem{lema}{Lema}
\newtheorem{definicija}{Definicija}
\newtheorem{pripomba}{Pripomba}
\newtheorem{primer}{Primer}
\newtheorem{zgled}{Zgled}
\newtheorem{zgledi}{Zgledi uporabe}
\newtheorem{zglediaf}{Zgledi aritmetičnih funkcij}
\newtheorem{oznaka}{Oznaka}

\title{Delitev brez zavisti}
\author{Tjaša Bajc}
\date{19.\ maj 2017}

\begin{document}


%%%%%%%%%%%%%%%%%%%%%%%%%%%%%%%%%%%%%%%%%%%%%%%%%%%%%%%%%%%%%%%%%%%%%

\begin{frame}
\titlepage
\end{frame}

%%%%%%%%%%%%%%%%%%%%%%%%%%%%%%%%%%%%%%%%%%%%%%%%%%%%%%%%%%%%%%%%%%%%%

\begin{frame}
\frametitle{Uvod}

Potrebno je privzeti naslednje:

\begin{itemize}
\item za vse igralce je mera celote $C$ enaka 1, $\mu_i(C) = 1, i \in \{1, 2, \ldots, n\}$;
\item kosov z enako mero med seboj ne razlikujemo; 
\item vsak kos je mogoče nadalje razdeliti na $k$ enako velikih kosov;
\item za vsaka kosa $A$ in $B$ je mogoče kos $A$ obrezati tako, da bo $\mu(A) = \mu(B)$, ali obratno.							
\end{itemize}

\end{frame}

%%%%%%%%%%%%%%%%%%%%%%%%%%%%%%%%%%%%%%%%%%%%%%%%%%%%%%%%%%%%%%%%%%%%%

\begin{frame}
\frametitle{Kaj pomeni ``pravična'' delitev?}

\begin{definicija}
Naj bo $C$ množica, $\mu_1, \mu_2, \ldots, \mu_n$ po vrsti mere igralcev $1, 2, \ldots, n$ in $\{P_1, P_2, \ldots, P_n\}$ delitev celote $C$ na $n$ delov, kjer igralcu $i$ pripada kos $P_i$. Delitev je:

\begin{itemize}
\item {\em sorazmerna\/}, če za vsak $i$ velja $\mu_i(P_i) \geq 1/n$;
\item {\em brez zavisti\/}, če za vsaka $i, j$ velja $\mu_i(P_i) \geq \mu_i(P_j)$;
\item{\em super brez zavisti\/}, če za vsaka $i, j$ velja $\mu_i(P_i) > 1/n$ in hkrati $\mu_i(P_j) < 1/n$ .								
\end{itemize}

\end{definicija}

\end{frame}

%%%%%%%%%%%%%%%%%%%%%%%%%%%%%%%%%%%%%%%%%%%%%%%%%%%%%%%%%%%%%%%%%%%%%

\begin{frame}

\frametitle{Protokoli}

Protokol je sestavljen iz

\begin{itemize}
\item pravil,
\item strategij,
\item dokaza.
\end{itemize}

\begin{zgled}

Oglejmo si metodo ``Reži in izberi'' za delitev med dva igralca.
\begin{enumerate}

\item Prvi igralec razdeli torto na dva dela \textcolor{gray}{(ki sta zanj enako velika)}.

\item Igralca si razdelita kose. Prvi izbere drugi igralec, prvi pa dobi kos, ki ga drugi ni izbral.

\end{enumerate}

{\em Komentar:} Ta delitev je {\em sorazmerna\/} in hkrati {\em brez zavisti\/}. Prvi igralec je dobil natančno $1/2$, drugi pa kvečjemu več. Če namreč njuni meri sovpadata, $\mu_1 = \mu_2$, je prav tako dobil natančno $1/2$, v nasprotnem primeru pa si lahko izbere večjega od obeh kosov.


\end{zgled}

\end{frame}


\end{document}